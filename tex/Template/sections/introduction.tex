\section{Introduction}
\label{sec:introduction}

Modeling time series data with irregular and sporadic observations is a 
persistent challenge in machine learning, with critical applications in domains such as healthcare, 
finance, and climate science. 
Traditional Recurrent Neural Networks (RNNs) and their variants, 
like LSTMs and GRUs, are designed for sequences with fixed, 
regular time steps and struggle to naturally handle the continuous-time nature of 
real-world processes.

Neural Ordinary Differential Equations (NODEs) \cite{chen2019neuralordinarydifferentialequations} 
emerged as a powerful paradigm to address this limitation. 
By parameterizing the derivative of a hidden state with a neural network, NODEs learn a continuous-time dynamical 
system directly from data, allowing for evaluation at any point in time. 
The GRU-ODE-Bayes model \cite{debrouwer2019gruodebayescontinuousmodelingsporadicallyobserved} 
builds upon this foundation, proposing a sophisticated architecture that propagates a system's 
hidden state using an ODE solver between observations and applies discrete, 
GRU-inspired updates at the moments observations are available.

While theoretically elegant, the practical implementation and performance characteristics of such 
complex models are not widely documented. 
The journey from a research paper's mathematical formulation to a working, efficient implementation is often 
fraught with non-trivial engineering challenges. 
This paper presents a deep dive into this process for the GRU-ODE-Bayes model. 
We document our journey from a naive implementation to a robust, vectorized model, 
tackling issues of data imbalance and computational bottlenecks.

Our investigation, conducted on three real-world physiological time series datasets, yields surprising and critical insights.
We find that the computational overhead of the core ODE solver component makes the model train significantly 
faster on a CPU than on a high-end GPU. 
More importantly, we demonstrate that a simplified baseline model, which removes the continuous-time ODE dynamic 
entirely, is not only orders of magnitude faster but also dramatically more accurate for the task of arrhythmia classification. 
This work aims to bridge the gap between theory and practice, providing a critical analysis of the GRU-ODE-Bayes
model's effectiveness and highlighting the crucial need to benchmark complex architectures
against strong, simple baselines.