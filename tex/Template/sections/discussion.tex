\section{Discussion}
\label{sec:discussion}

Our findings present a cautionary tale about the application of complex models. 
The GRU-ODE-Bayes architecture, while theoretically compelling, failed to deliver on its promise for the practical 
task of arrhythmia classification on our datasets. 
The significant computational overhead and inferior performance compared to a simple baseline suggest that the 
model's core assumption—that modeling the continuous dynamics between heartbeats is beneficial—does not hold 
true for this problem. 
The classification task may depend more on the sequence of discrete QRS events rather 
than the subtle continuous evolution between them.

The performance anomaly where a CPU outperforms a GPU also serves as a critical reminder that hardware acceleration 
is not a universal solution. 
The nature of the computational workload is paramount, and algorithms involving many small, 
sequential steps may not benefit from a GPU's massively parallel architecture.